\subsection{Configure the Storage Virtual Appliance for vSphere}

The vSphere Storage Appliance is new in vSphere 5.\\

The VSA is not included with any version of vSphere. It costs \$5995 per
instance and can use up to 3 nodes. vSphere Essentials Plus with the VSA
costs \$7995.

\subsubsection{Define Storage Virtual Appliance (SVA) architecture}

The VSA provides NFS storage using the local storage across up to 3 ESXi
servers.\\

Each ESXi server has a special VSA VM vm on it. There is a 1:1 relationship
between vCenter and VSA and they must be on the same subnet.

The VSA is fully redundant.\\

VSA uses RAID 1 across nodes and RAID 10 within each node.\\

If there are two members in the VSA cluster, the VSA cluster service acts
as a third member in case one fails. If there are three members in the cluster,
the VSA cluster service is not needed.

\subsubsection{Configure ESXi hosts as SVA hosts}

The ESXi servers must be a fresh install.\\

Multiple NICs are highly recommended.\\

vCenter must not be running as a VM on the ESXi hosts in the cluster.\\

There cannot be any VMs running on the ESXi hosts during VSA installation.\\

There can only be one datastore on the ESXi host and it must be a local disk
(there cannot be any SAN connections on the server).\\

None of the ESXi hosts used in a VSA cluster can be in an HA cluster.

\subsubsection{Configure the storage network for the SVA}

\subsubsection{Deploy/Configure the SVA Manager}

\subsubsection{Administer SVA storage resources}

VSA san maintenance mode options:

\begin{itemize}
\item entire VSA cluster
\item single VSA node
\end{itemize}

If a node is taken out of the cluster, changed blocks are tracked until the
node is added again. If a new node is added a full sync is done.

\subsubsection{Determine use case for deploying the SVA}

Shared storage (SAN or NAS) is required to implement many of VMware's core
features like vMotion, VMHA, and DRS. Due to high costs, many companies have
been unable to implement these features. The VSA will allow all customers to
affordably implement a SAN and use advanced vSphere features.\\

VSA requires no additional hardware (CAPEX savings) and requires no dedicated
SAN administration (OPEX savings).

\subsubsection{Determine appropriate ESXi host resources for the SVA}

Only a short list of physical servers are supported by VMware.\\

Must have 6GB of ram minimum per server.\\

Each ESXi host must have a RAID controller that supports RAID10.\\

All ESXi servers must have the same hardware configuration.\\
