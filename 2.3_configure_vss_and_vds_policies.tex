\subsection{Configure vSS and vDS policies}

\subsubsection{Identify common vSS and vDS policies}

\subsubsection{Configure dvPort group blocking policies}

\subsubsection{Configure load balancing and failover policies}

To avoid losing connectivity during failover or failback, put your physical
switch in portfast or portfast trunk mode.\\

Load balancing can be based on:

\begin{itemize}
\item port id of virtual port the traffic entered switch on
\item hash of source and destination IP addresses
\item hash of source MAC
\item physical NIC load
\item explicit order, use highest order uplink from list
\end{itemize}

IP-based teaming requires that the physical switch be configured with
etherchannel. When using other options etherchannel should be disabled.\\

When using iSCSI multipathing, the VMkernel interface must have one active
adapter and no standby adapters.\\

When using IP-hash load balancing do not configure standby links.\\

Failover detection can be done in two ways:

\begin{itemize}

\item link status only, detects cable pulls and power failures, but not
configuration errors such as being blocked by spanning tree or set to wrong
VLAN

\item beacon probing, do not use with IP-hash load balancing

\end{itemize}

\subsubsection{Configure VLAN settings}

802.1Q is the VLAN standard.\\

VLAN 4095 can see traffic from any VLAN.\\

VLANs can be configured using the following methods:

\begin{itemize}

\item External Switch Tagging (EST), physical switch does all VLAN tagging,
virtual switch port groups have VLAN set to 0

\item Virtual Switch Tagging (VST). VLAN tagging performed by virtual switch
before leaving the host, host adapters are connected to trunk ports on
physical switch, port groups have VLAN set

\item Virtual Guest Tagging (VGT), VLAN tagging performed by virtual machine,
physical switch ports are set to trunk, requires driver on virtual machine

\end{itemize}

To use PVLANS, the physical switch connected to the host needs to support them.

\subsubsection{Configure traffic shaping policies}

ESXi shapes outbound network traffic on standard switches and input and
outbound traffic on distributed switches.

\subsubsection{Enable TCP Segmentation Offload support for a virtual machine}

TCP Segmentation offload (TSO) allows a TCP/IP stack to emit large frames
up to 64K which are broken down into MTU-sized frames by the network adapter.

\subsubsection{Enable Jumbo Frames support on appropriate components}

Jumbo frames up to 9K are supported. The physical adapter must support jumbo
frames.

\subsubsection{Determine appropriate VLAN configuration for a vSphere implementation}
