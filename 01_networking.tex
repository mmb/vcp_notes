\documentclass{article}

\begin{document}

\section{Networking}

802.1Q is the VLAN standard.\\

TCP Segmentation offload (TSO) allows a TCP/IP stack to emit large frames
up to 64K which are broken down into MTU-sized frames by the network adapter.

\subsection{vSphere Standard Switches}

The default number of logical ports for a standard switch is 120.\\

All portgroups in the same broadcast domain are given the same network label.
If two portgroups are not in the same broadcast domain they will have different
network labels.\\

VLAN 4095 can see traffic from any VLAN.

\subsection{VMkernel Ports}

Can be used for:

\begin{itemize}
\item Fault Tolerance logging
\item IP storage (iSCSI, NFS)
\item management
\item vMotion
\end{itemize}

Each VMkernel port has an IP address configured for it or can use DHCP.\\

You can enable vMotion and IP storage for only one port group per host.

\subsection{vSphere Distributed Switches}

\subsection{Private VLANs}

To use PVLANS, the physical switch connected to the host needs to support them.

\subsection{Jumbo Frames}

Jumbo frames up to 9K are supported. The physical adapter must support jumbo
frames.

\subsection{Failover and Load Balancing Policy}

To avoid losing connectivity during failover or failback, put your physical
switch in portfast or portfast trunk mode.

\subsubsection{Load Balancing Options}

Based on:

\begin{itemize}
\item port id of virtual port traffic entered switch on
\item hash of source and destination IP addresses
\item hash of source MAC
\item physical NIC load
\item explicit order, use highest order uplink from list
\end{itemize}

IP-based teaming requires that the physical switch be configured with
etherchannel. When using other options etherchannel should be disabled.

\subsubsection{Failover Detection}

\begin{itemize}

\item link status only, detects cable pulls and power failures, but not
configuration errors such as being blocked by spanning tree or set to wrong VLAN

\item beacon probing, do not use with IP-hash load balancing

\end{itemize}

When using iSCSI multipathing, the VMkernel interface must have one active
adapter and no standby adapters.\\

When using IP-hash load balancing do not configure standby links.

\subsection{Traffic Shaping}

ESXi shapes outbound network traffic on standard switches and input and
outbound traffic on distributed switches.

\subsection{Monitoring}

NetFlow settings are configured at the vSphere distributed switch level.

\subsection{VLAN Configuration}

VLANs can be configured using the following methods:

\begin{itemize}

\item External Switch Tagging (EST), physical switch does all VLAN tagging,
virtual switch port groups have VLAN set to 0

\item Virtual Switch Tagging (VST). VLAN tagging performed by virtual switch
before leaving the host, host adapters are connected to trunk ports on
physical switch, port groups have VLAN set

\item Virtual Guest Tagging (VGT), VLAN tagging performed by virtual machine,
physical switch ports are set to trunk, requires driver on virtual machine

\end{itemize}

\subsection{Port Mirroring}

If you don't select ``Allow normal IO on destination ports'' mirrored traffic
is allowed out but not in on destination ports.

\subsection{MAC Addresses}

When manually assigning MACs the fourth octet must be between 00 and 3F to
avoid conflicts with VMware Workstation and Server MAC addresses.\\

Generated MACs are constructed using the VMware OUI, the SMBIOS UUID of the
host and hash of the entity name.

\subsection{Best Practices}

\begin{itemize}

\item separate network services for security and performance

\item put vMotion on its own network using VLANs or a separate physical network
(preferred)

\item when using passthrough devices on Linux 2.6.20 or earlier avoid MSI
and MSI-X due to performance problems

\item create a standard or distributed switch for each network service, or
at least put them in port groups in different VLANs

\item as long as there is one uplink available, virtual machines and network
services will be able to connect to the physical network

\item deploy firewalls on virtual machines that route between networks with
uplinks to physical networks and networks with no uplinks

\item use vmxnet3 virtual NICs

\item every physical NIC connected to the same vSphere standard or distributed
switch should be on the same physical network

\item set all VMkernel network adapters to the same MTU

\end{itemize}
\end{document}
