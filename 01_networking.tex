\documentclass{article}

\begin{document}

\section{Networking}

802.1Q is the VLAN standard.\\

TCP Segmentation offload (TSO) allows a TCP/IP stack to emit large frames
up to 64K which are broken down into MTU-sized frames by the network adapter.

\subsection{vSphere Standard Switches}

The default number of logical ports for a standard switch is 120.\\

All portgroups in the same broadcast domain are given the same network label.
If two portgroups are not in the same broadcast domain they will have different
network labels.\\

VLAN 4095 can see traffic from any VLAN.

\subsection{VMkernel Ports}

Can be used for:

\begin{itemize}
\item Fault Tolerance logging
\item IP storage (iSCSI, NFS)
\item management
\item vMotion
\end{itemize}

Each VMkernel port has an IP address configured for it or can use DHCP.\\

You can enable vMotion and IP storage for only one port group per host.

\subsection{vSphere Distributed Switches}

\subsection{Private VLANs}

To use PVLANS, the physical switch connected to the host needs to support them.

\subsection{Jumbo Frames}

Jumbo frames up to 9K are supported. The physical adapter must support jumbo
frames.

\end{document}
