\subsection{Configure Shared Storage for vSphere}

vSphere provides host-level storage virtualization, which logically abstracts
the physical storage layer from virtual machines.

\begin{itemize}

\item VMs are not aware of storage

\item VMs use virtual disks

\item virtual disks are easier to manage

\item VMs use virtual SCSI controllers to access virtual disks that are on
VMFS datastores, NFS datastores or raw disk mapping (RDM)

\end{itemize}

ESXi storage options:

\begin{itemize}
\item local disk with VMFS datastore
\item SAN - iSCSI software, iSCSI hardware, Fibre Channel
\item NAS - NFS
\item VSA - vSphere Storage Appliance
\end{itemize}

\begin{description}

\item[block level]
presentation of storage that can be formatted and managed as a local drive,
for example, a Windows host that has been presented with a unit of storage
from a SAN views the storage as if it were a local disk and the host
is permitted to format the disk, includes iSCSI, FC and local disk

\item[file level transfer]
presentation of storage that has been formatted and is managed by the host
presenting the storage, for example, a Windows host with a mapped drive to a
shared directory does not have the ability to format the drive, includes
NFS and CIFS

\end{description}

\begin{tabular}{ | c | p{2cm} | p{2cm} | c | c | c | }
\hline
& Protocol & Hardware & Data Transfer & Performace & Cost \\ \hline
NAS & SMB/FS over TCP/IP & NIC (1G or 10G) & file level & low to high & low \\ \hline
iSCSI & SCSI over TCP/IP & NIC or iSCSI HBA (1G or 10G) & block level & medium to high & medium \\ \hline
fibre channel & SCSI over Fiber Channel & Fibre Channel HBA (2, 4 or 8G) & block level & high & high \\ \hline
\end{tabular}\\

SAN could be iSCSI or Fibre Channel. Fibre Channel SAN packages SCSI commands
into FC frames. The servers connects to the SAN using an HBA (similar to a
NIC). The HBA connects to a Fibre Channel switch (known as a fabric). The FC
switch connects to a storage processor (SP).\\

SAN Benefits

\begin{itemize}
\item maintenance with zero downtime
\item load balancing with vMotion and svMotion
\item storage consolidation and central management
\item disaster recovery
\item array migrations and storage updates
\item use of advanced features
\end{itemize}

\subsubsection{Identify storage adapters and devices}

Storage adapter types:

\begin{itemize}
\item SCSI
\item iSCSI
\item RAID
\item Fiber Channel
\item Fiber Channel over Ethernet (FCoE)
\item Ethernet
\end{itemize}

\begin{description}

  \item[extent] a disk area that can be added to a datastore

  \item[HBA] host bus adapter, storage interface card

  \item [LUN] logical unit number, a device addressed by SCSI, iSCSI or Fiber
    Channel, logical disk, aggregate of physical disks, can be a RAID group

  \item[RDM] raw device mapping, special type of storage disk where ESXi
    controls disk access, a mapping file on VMFS acts as a proxy, used when
    applications need to access raw LUNs (Microsoft Clustering), think of
    as a symbolic link from VMFS to LUN

  \item[spanned volume]
    dynamic volume spread across a number of extents

  \item[WWN] world-wide name, uniquely identifies a Fiber Channel adapter,
    like a MAC address, 64-bit address

  \item[target] a single storage unit that the host can access

  \item[NPIV]
    N-Port ID Virtualization, allows a single FC HBA port to register with the
    fabric using multiple WWPNs, multiple FC initiators can share a single port,
    NPIV is used only for VMs with RDM disks, used for fine-grained control
    at the VM level, NPIV does not support storage vMotion

\end{description}

RDM has two modes:

\begin{description}

\item[virtual compatibility]
RDM acts like virtual disk file, supports snapshots

\item[physical compatibility]
lower level, allows direct access to the SCSI device
 
\end{description}

ESXi natively supports a number of brands of Fibre Channel HBAs.

Fibre Channel Storage Array Types

\begin{description}

\item[active-active]
all paths can be active at the same time, access to LUNs through all storage
ports at the same time

\item[active-passive]
one storage processor at a time is actively providing access to a LUN

\item[asymmetrical]
supports Asymmetric Logical Unit Access (ALUA), provides different levels of
access per port, hosts determine states of targets ports and prioritize paths

\end{description}

\subsubsection{Identify storage naming conventions}

Fibre Channel SAN LUNs are addressed using a 4-part address:

\begin{verbatim}
vmhba#:storage processor#:LUN#:PARTITION#
\end{verbatim}

vmhba0:0:1 is vmhba0 / SP 0 / LUN 1.\\

An address is like a route in this case.\\

iSCSI uses IQN (iSCSI qualified name) to identify SCSI targets and initiators.
It is the equivalent of WWN in Fibre Channel.\\

IQN format:\\

\begin{verbatim}
year-month.reversed domain:organizational unique id
\end{verbatim}

The date is the date the domain was registered.\\

Some storage vendors present a single target with multiple LUNs, some present
multiple targets with one LUN each.\\

If a LUN is shared by multiple hosts, it must be presented to all hosts with
the same UUID.\\

Runtime name is not a reliable identifier for a device and is not persistent.

\subsubsection{Identify hardware/dependent hardware/software iSCSI initiator requirements}

\begin{description}
\item[iSCSI]
internet small computer system interface, uses IP network
\end{description}

iSCSI is cheaper than Fibre Channel in many ways (components, network
infrastructure). It can used existing hardware and supports almost all vSphere
features.\\

iSCSI it not as reliable as Fibre Channel and does not perform as well.\\

An ESXi server can have an iSCSI hardware initiator (a special iSCSI NIC
installed in the server) or a software initiator which is all in software and
uses the existing NIC. You can boot from SAN when using iSCSI only if using a
hardware HBA.\\

iSCSI SANs call the disk array a target because it can be any storage attached
to a server, not necessarily a disk array.\\

Dependent hardware iSCSI depends on VMware networking and interfaces provided
by VMware.

\subsubsection{Compare and contrast array thin provisioning and virtual disk thin provisioning}

Thin provisioned virtual disks start small and grow as needed. Array thin
provisioning is the same concept applied to LUNs. The hosts see the LUNs as
their full logical size.

\subsubsection{Describe zoning and LUN masking practices}

Zoning is the grouping of HBAs and giving the groups access to certain LUNs.
A zone is a set of HBA WWNs.\\

Zoning is the process of allowing ESXi hosts to communicate with a storage
device through the Fiber Channel switched fabric.\\

Zoning is performed from the management interface of the Fibre Channel Switch.\\

A relationship between the ESXi hosts and LUNs is created through storage
groups.\\

For ESXi hosts the preferred zoning is single-initiator-single-target.

\begin{description}

\item[soft zoning]
SP controls the visibility of LUNs to HBAs

\item[hard zoning]
occurs at FC switch level, controls of visibility of SPs on a per-port basis

\item[LUN masking]
can occur at SP level or at the host level (for example, make host esx01 not be
able to see luns 3 and 4)

\end{description}

\subsubsection{Scan/Rescan storage}

\subsubsection{Identify use cases for FCoE}

Converged Network Adapters (CNA) contain network and Fibre Channel
functionality on the same hardware.\\

ESXi can do software FCoE but requires a NIC with Data Center Bridhing (DCB)
and I/O offloading capabilites.\\

On switch ports used for FCoE that communicate with ESXi hosts, disable
spanning tree protocol. It could delay FCoE initiation and cause an all
path down condition.

\subsubsection{Create an NFS share for use with vSphere}

The NFS client built into ESXi uses NFS version 3.\\

In vSphere 5, Storage IO Control supports NFS.

\subsubsection{Connect to a NAS device}

NAS is network attached storage. ESXi can use NFS NAS.

\subsubsection{Enable/Configure/Disable vCenter Server storage filters}

Storage filters help avoid storage device corruption or performance
degradation that can be caused by an unsupported use of storage devices.\\

For example, not showing LUNs that already have VMFS filesystems on them
as candidates to be formatted.

\subsubsection{Configure/Edit hardware/dependent hardware initiators}

\subsubsection{Enable/Disable software iSCSI initiator}

\begin{enumerate}
\item enable software iSCSI in the host firewall
\item add storage adapter to host
\end{enumerate}

\subsubsection{Configure/Edit software iSCSI initiator settings}

You can alias iSCSI software initiators with your own names.

Making iSCSI discover LUNs available to the target can be done statically
or dynamically:

\begin{description}

\item[static]
provide ip address of iSCSI server and target name

\item[dynamic]
provide ip address of iSCSI server, discovers targets, also called SendTargets,
you cannot discover LUNs with a LUN ID greater than 255

\end{description}

You need to bind one or more physical NICs to they can be used to communicate
with the target device.\\

\subsubsection{Configure iSCSI port binding}

\subsubsection{Enable/Configure/Disable iSCSI CHAP}

CHAP (Challenge Handshake Authentication Protocol) is used to secure iSCSI
connections using a CHAP name and secret. Mutual CHAP, where the initiator and
target both authenticate each other is supported for software and dependent
hardware iSCSI. In normal CHAP the target authenticates the host only.

\subsubsection{Determine use case for hardware/dependent hardware/software iSCSI initiator}

ESXi does not support multipathing when you combine an independent hardware
adapter with software iSCSI or dependent iSCSI on the same host.

\subsubsection{Determine use case for and configure array thin provisioning}

Traditional LUNs are thick provisioned. When a LUN is thin-provisioned, the
storage array reports reports the LUN's logical size which might be larger
than the real physical capacity backing it.\\

A VMFS datastore backed by a thin-provisioned LUN sees only the logical
size.\\

Using array integration APIs the host can monitor the use of space on
thin-provisioned LUNs. The array can warn the host if space is low. The
host can inform the array about freed datastore space so the array can reclaim
it.\\

For array thin provisioning ESXi 5.0 or later is required. The storage array
must have firmware that suppors T10 Storage APIs.
