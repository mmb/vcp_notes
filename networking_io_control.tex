\section{Network I/O Control}

Network I/O control provides quality of service for network traffic by type.
It requires vSphere Enterprise Plus licensing and must be enabled on a
virtual distributed switch.\\

Network I/O control was a new feature in vSphere 4.1.\\

Network I/O control uses shares similar to the CPU and memory shares used
by VMs and resource pools. It ensures the best performance for critical
vSphere infrastructure such as NFS, iSCSI, FT and vMotion when there is
high network load.

\subsection{New in vSphere 5}

\begin{itemize}
\item user-defined network resource pools
\item multi-tenant deployment
\item bridges virtual and physical QOS with per resource pool 802.1p tagging
\end{itemize}

\subsection{vSphere Client}

In the vSphere client, network resource pools are associated with the virtual
distributed switch. User-defined network resource pools can be created,
have shares set on them and then be associated with distributed virtual
portgroups.
