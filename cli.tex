\section{vSphere Command Line interface Options}

Reasons for using CLI:

\begin{itemize}
\item faster once you overcome the learning curve
\item automation
\item integration with other products
\end{itemize}

\subsection{ESXi Tech Support Mode}

ESXi Tech Support Mode is a CLI in ESXi accessible on the console or over ssh
(Remote Tech Support Mode).\\

Tech Support Mode or Remote Tech Support Mode can be enabled on the ESXi
server console using the DCUI or using the vSphere client.

Alt-F1 opens the ESXi console. Alt-F2 returns to the DCUI.

\subsection{PowerCLI and Project Onyx}

PowerCLI is a PowerShell-based CLI that runs on Windows. Quest PowerGUI is
GUI front-end for PowerCLI that provides pre-created scripts.\\

Project Onyx acts as a proxy between the vSphere Client and vCenter Server
and can intercept vSphere Client commands and show them as SOAP or PowerCLI
commands.

\subsection{vSphere CLI}

VCLI is a Windows Application that allows esxcfg and esxcli commands to be
run in the Windows command prompt.

\subsection{vSphere Management Assistant}

vSphere Management Assistant is a Linux-based virtual appliance that has
the CLI installed in it. It has the esxcfg and esxcli commands and fastpass
authentication that allows you to authenticate once and perform multiple
commands.\\

The login is vi-admin.

\subsection{ESXi Tech Support Mode Commands}

Traditional ESXi configuration commands run from Tech Support Mode start with
with \texttt{esxcfg-}. These commands have been deprecated and you should
use the \texttt{esxcli} command instead. \texttt{esxcli} is new in vSphere
v5.\\

Example ESXi console commands:

\begin{verbatim}
esxcfg-nics
esxcli vm list
esxcli network ip interface list
\end{verbatim}
