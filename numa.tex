\section{NUMA}

NUMA (non-uniform memory access) systems are advanced server platforms with
more than one system bus. They can use a lot of processors in a single system
with a good price to performance ratio.\\

NUMA was developed to solve the problem of the high memory bandwidth
requirements of high-speed SMP processors. NUMA system have several nodes each
containing its own processors and memory. Each node can use the memory of
other nodes over a NUMA connection (with higher latency than using its own
memory).\\

ESXi has a NUMA scheduler that assigns a home NUMA node to each virtual
machine and prefers to give the VM memory in its home node. The virtual CPUs
of the VM are only allowed to run on the home node's processors. The NUMA
scheduler can change a virtual machine's home node to respond to changes
in system load.\\

ESXi 5.0 can expose virtual NUMA topology to guest operating systems.\\

Transparent page sharing is optimize for NUMA.\\

Virtual NUMA is enabled by default on VMs with more than 8 vCPUs. If
the VM's cores per socket is greater than one and the number of virtual cores
is greater than 8 the virtual NUMA node size matches the virtual socket size.
If not, virtual NUMA nodes are create to match the topology of the first
physical host where the VM is powered on.

