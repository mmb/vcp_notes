\section{Storage I/O Control}

Most virtualization performance issues are caused by latency in shared
storage. Storage I/O control provides quality of service to ensure that all
VMs get the storage performance they need.\\

Storage I/O control is necessary so that one VM doesn't slow down another's
use of storage resources.\\

Storage I/O control uses the concept of shares, similar to the way VMs
and resource pools use shares for CPU and memory.\\

Storage I/O Control is enabled is enabled per datastore. Once enabled, ESXi
servers monitor the latency of that datastore. It is enabled by default on
Storage DRS-enabled datastore clusters.\\

The congestion threshold is the latency limit that signals congestion and
causes shares to be used. All datastores that share the same spindles should
have the same congestion threshold.

\subsection{Requirements}

\begin{itemize}
\item datastores must be managed by a single vCenter server
\item FC, iSCSI and NFS are supported, RDM is not
\item support for NFS is new in vsphere 5
\item datastores with multiple extents are not supported
\item vsphere 4.1 or later (5 to use NFS)
\end{itemize}

\subsection{Enabling SIOC}

\begin{enumerate}
\item go to ESXi server
\item Configuration tab
\item Storage
\item select Datastore
\item select Properties
\end{enumerate}

The number of shares and maximum IOPS can be set per VM. The number of shares
defaults to normal (1000). To limit a VM based on Mb/s, you can convert into
IOPS based on the typical I/O size for that VM. You must set a limit on all
the disks of a virtual machine to use limits.\\

Shares are set per-disk on a VM and are all disks are added to get the VM's
total IO shares.
