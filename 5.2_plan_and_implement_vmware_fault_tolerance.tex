\subsection{Plan and Implement VMware Fault Tolerance}

FT provides a higher level of availability than HA. FT creates and maintains
a secondary VM that is identical to and continuously available to replace
the primary VM if it fails. With FT there is no loss of data, transactions or
connections when a failure occurs.\\

The primary and secondary in an FT pair are identical at any point in
instruction execution. Using VMware Lockstep, the VMs execute identical x86
instructions.\\

FT logging traffic is not encrypted.\\

The primary and secondary VMs exchange heartbeats. If the primary fails,
the secondary is activated and a new secondary is created.\\

The primary and secondary cannot run on the same host.\\

FT can be used with DRS when EVC is enabled. This gives FT machines better
initial placement and includes them in the cluster's load balancing
calculations.\\

DRS will not place more than four FT VMs on a host. This is configurable.\\

\subsubsection{Identify VMware Fault Tolerance requirements}

FT works even if vCenter is unavailable.

Cluster Requirements

\begin{itemize}

\item HA-enabled cluster

\item host certificate checking must be enabled

\item at least two FT-certified hosts running the same FT version

\item hosts must have access to the same datastores and networks

\item FT logging and vMotion networks must be configured (3 1GB nics minumum
are recommended)

\end{itemize}

Host Requirements

\begin{itemize}
\item FT-compatible processors
\item licensed and certified for FT (vSphere Enterprise or Enterprise Plus)
\item hardware virtualization enabled in the BIOS
\item ESXi servers must be running same build
\end{itemize}

Virtual Machine Requirements

\begin{itemize}
\item no unsupported devices attached
\item must use virtual RDM or VMDK disks that are thick provisioned
\item incompatible features must not be running
\item must be on shared storage
\item must only have 1 vCPU
\item guest OS must be supported, some require shutdown to enable
\end{itemize}

vSphere features not supported with FT:

\begin{itemize}
\item snapshots
\item storage vMotion
\item linked clones
\item backups with VMware Data Recovery, etc.
\end{itemize}

VM features not supported with FT:

\begin{itemize}
\item SMP processors
\item physical RDM
\item CD-ROM or floppy backed by a physical or remote device
\item paravirtualized guests
\item USB and sound devices
\item NPIV
\item NIC passthrough
\item vlance NIC drivers
\item think provisioned disks
\item hot-plugging devices
\item EPT/RVI (extended page tables / rapid virtualization indices)
\item serial and parallel ports
\item IPv6
\item video cards with 3D enabled
\end{itemize}

To change the network a NIC on an FT VM is plugged into, FT must be disabled
first.\\

When FT is turned on, the VM's memory reservation is set to its memory size
and its memory limit is unset. Memory reservation, size, limit and shared
cannot be changed while FT is on.\\

VMware Site Survey is a free vSphere plugin that can determine if FT can be
used on a cluster.

\subsubsection{Configure VMware Fault Tolerance networking}

\subsubsection{Enable/Disable VMware Fault Tolerance on a virtual machine}

\subsubsection{Test an FT configuration}

\subsubsection{Determine use case for enabling VMware Fault Tolerance on a virtual machine}

\begin{itemize}

\item application with long-lasting client connections that need to be
maintained

\item custom applications with no other way of doing clustering

\item to avoid complicated custom clustering solutions

\end{itemize}

Temporary on-demand fault tolerance could be used during critical time period.
