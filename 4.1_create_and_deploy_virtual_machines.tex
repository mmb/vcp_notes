\subsection{Create and Deploy Virtual Machines}

\subsubsection{Identify capabilities of virtual machine hardware versions}

\subsubsection{Identify VMware tools device drivers}

\subsubsection{Identify methods to access and use a virtual machine console}

\subsubsection{Identify virtual machine storage resources}

\subsubsection{Place virtual machines in selected ESXi hosts/Clusters/Resource Pools}

\subsubsection{Configure and Deploy a Guest OS into a new virtual machine}

\subsubsection{Configure/Modify disk controller for virtual disks}

\subsubsection{Configure appropriate virtual disk type for a virtual machine}

\subsubsection{Create/Convert thin/thick provisioned virtual disks}

\subsubsection{Configure disk shares}

Disk shares are relevant only within a given host.

\subsubsection{Install/Upgrade/Update VMware Tools}

\subsubsection{Configure virtual machine time synchronization}

\subsubsection{Convert a physical machine using VMware Converter}

\subsubsection{Import a supported virtual machine using VMware Converter}

\subsubsection{Modify virtual hardware setting using VMware Converter}

\subsubsection{Configure/Modify virtual CPU and Memory resources according to OS and application requirements}

\subsubsection{Configure/Modify virtual NIC adapter and connect virtual machines to appropriate network resources}

When manually assigning MACs the fourth octet must be between 00 and 3F to
avoid conflicts with VMware Workstation and Server MAC addresses.\\

Generated MACs are constructed using the VMware OUI, the SMBIOS UUID of the
host and hash of the entity name.

\subsubsection{Determine appropriate datastore locations for virtual machines based on application workloads}
