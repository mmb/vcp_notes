\subsection{Create and Deploy Virtual Machines}

There are three ways to create a virtual machine:

\begin{itemize}
\item fresh OS install
\item download and deploy a virtual appliance (www.vmware.com/appliances)
\item P2V conversion
\end{itemize}

Virtual appliances can be downloaded using a web browser or directly from the
vSphere Client by going to File / Browse VA Marketplace.\\

When creating a lot of VMs, keep in mind:

\begin{itemize}
\item cost of license
\item management overhead
\item documentation
\item training
\item backups
\item 3rd party software licenses based on the number of servers
\end{itemize}

\subsubsection{Identify capabilities of virtual machine hardware versions}

vSphere 5 uses VM hardware version 8, which supports:

\begin{itemize}
\item up to 32 vCPUs, including multicore (cores vs sockets)
\item up to 1TB RAM
\item up to 1 million IOPS
\item client connected USB (from client running vSphere client)
\item USB 3.0
\item non hardware-accelerated 3D graphics support
\item UEFI virtual BIOS (look up UEFI)
\end{itemize}

\subsubsection{Identify VMware tools device drivers}

VMware tools contains the following drivers:

\begin{itemize}
\item SVGA
\item VMXNet network driver for some guest OSs
\item BusLogic SCSI for some guests
\item memory control driver for efficient memory allocation between VMs
\item sync driver to quiesce IO for backup
\item VMware mouse driver
\end{itemize}

\subsubsection{Identify methods to access and use a virtual machine console}

Always install VMware tools on guests.\\

Console hot keys:

\begin{description}

\item[Ctrl-Alt]
release mouse (when VMware tools are not installed)

\item[Ctrl-Alt-Insert]
send Ctrl-Alt-Delete to the guest

\item[Ctrl-Alt-Enter]
switch back and forth to full screen

\end{description}

\subsubsection{Identify virtual machine storage resources}

\subsubsection{Place virtual machines in selected ESXi hosts/Clusters/Resource Pools}

\subsubsection{Configure and Deploy a Guest OS into a new virtual machine}

To deploy a guest OS onto a new VM you need an ISO on one of the following:

\begin{itemize}
\item local VMFS datastore on server or SAN
\item physical media on ESXi server (CD drive)
\item client device, ISO file on you local computer or a network share
\end{itemize}

\subsubsection{Configure/Modify disk controller for virtual disks}

Set in VM / Edit Settings / Hardware.\\

Use VMware Paravirtual if there will be 2000 IOPS or more, otherwise use
LSI Logic SAS.

\subsubsection{Configure appropriate virtual disk type for a virtual machine}

\subsubsection{Create/Convert thin/thick provisioned virtual disks}

Virtual Disk Types

\begin{description}

\item[thick lazy zeroed]
default, all space is allocated when the virtual disk is created, zeroed on
demand when written, also known as a flat disk

\item[thick eager zeroed]
all space is allocated when the virtual disk is created, zeroed when created
(so creation takes longer), supports clustering features like fault tolerance

\item[thin]
disk starts small and uses more storage as needed, does not work with
clustering like FT

\end{description}

Virtual Disk Modes

\begin{description}

\item[independent persistent]
behave like conventional disks in physical computers, all data written to
disk is permanent

\item[independent nonpersistent]
change to disk are discarded when you power off or rest the VM, the VM restarts
in the same state every time

\end{description}

\subsubsection{Configure disk shares}

Disk shares are relevant only within a given host.

\subsubsection{Install/Upgrade/Update VMware Tools}

VMware tools are drivers, an application and configuration that are installed
after the guest OS is installed to enhance it. They provide an interface
between the guest and the virtualization layer. Without them you lose
important functionality and convenience.\\

To install or upgrade VMware tools go to VM / Guest / Install/Upgrade VMware
tools.\\

On the guest VMware tools appears as a simple application: VMwareService.exe
on Windows and vmware-guestd on Linux and Solaris.\\

VMware tools provide time-sync between the host and guest and on Windows
control grabbing and releasing the mouse when in the console.\\

VMware Tools control panel can be used to modify and shrink virtual disks and
connects and disconnect virtual devices.\\

Scripts can be configured to run when the power state of a VM changes.\\

The VMware user process (VMwareUser.exe on Windows and vmware-user on Linux
and Solaris) enables copy and paste of text between guest and host.\\

VMware Tools Installers are ISO images that are installed when ESXi is
installed.\\

Without VMware tools the guest shutdown and restart options in vSphere client
do not work.\\

Installing VMware tools in Linux:

\begin{itemize}
\item use rpm or tar installer (rpm is preferred)
\item for tar run vmware-install.pl
\item run vmware-config-tools.pl
\item tools are installed in /usr/lib/vmware-tools
\item configuration files are in /etc/vmware-tools
\item executables are in /usr/bin
\item to start vmware toolbox run vmware-toolbox
\end{itemize}

\subsubsection{Configure virtual machine time synchronization}

\subsubsection{Convert a physical machine using VMware Converter}

\subsubsection{Import a supported virtual machine using VMware Converter}

\subsubsection{Modify virtual hardware setting using VMware Converter}

\subsubsection{Configure/Modify virtual CPU and Memory resources according to OS and application requirements}

\subsubsection{Configure/Modify virtual NIC adapter and connect virtual machines to appropriate network resources}

When manually assigning MACs the fourth octet must be between 00 and 3F to
avoid conflicts with VMware Workstation and Server MAC addresses.\\

Generated MACs are constructed using the VMware OUI, the SMBIOS UUID of the
host and hash of the entity name.

\subsubsection{Determine appropriate datastore locations for virtual machines based on application workloads}
