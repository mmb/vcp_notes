\section{vSphere 5 vRAM Pooled Pricing}

vSphere 5 vRAM pooled pricing is a significant change. It may affect how
virtual machines are provisioned and cost.\\

vSphere is sold per CPU socket on the ESXi hosts. Each vSphere edition comes
with a vRAM entitlement per socket that is pooled per version of vCenter.\\

If the sum of configured vRAM of all powered on VMs is less than the sum of
all socket vRAM entitlement you are in compliance. Configured vRAM is
calculated as a 12-month average.\\

vSphere Desktop Edition is sold per VDI user desktop so the the pooled
vRAM limits do not apply to it.

\subsection{vRAM Entitlements}

\begin{description}
\item[Hypervisor Free] 32GB per server
\item[Essentials and Essentials Plus] 32GB per socket
\item[Standard] 32GB per soocket
\item[Enterprise] 64GB per socket
\item[Enterprise Plus] 96GB per socket
\end{description}

If the vRAM pool is exceeded:

\begin{itemize}

\item there is a soft limit on virtual machines (when using Standard licenses)

\item the EULA is violated and a warning is generated (with Standard,
Enterprise or Enterprise Plus licenses)

\item the EULA is violated and VMs will not power on (with Essentials and
Essentials Plus licenses)

\end{itemize}

\subsection{Ways to Prevent Non-compliance}

\begin{itemize}

\item don't overprovision memory

\item power off unused VMs

\item reduce application memory use

\item all hosts count toward the pool including HA hosts that are in standby
mode and remote servers if you are using linked vCenter

\end{itemize}

\subsection{Reporting}

vRAM pooled pricing reports can be viewed in vSphere client if the vSphere
web client server is installed. The vSphere 5 license validator script and
vSphere license advisor tools can also be used.
