\subsection{Configure vNetwork Distributed Switches}

The vSphere Distributed Switch creates a centralized virtual switch that
multiple ESXi hosts can subscribe to. It reduces networking configuration and
changes and allows you to centrally manage networking for VMs across multiple
ESXi hosts.\\

A distributed switch can have up to 30,000 ports.\\

Distributed switch give you consistent network configuration and stats as
VMs are migrated using vMotion.\\

Distributed virtual port groups are similar to standard vSwitch port groups
but are distributed across multiple ESXi hosts.\\

The distributed switch was improved in vSphere 5.\\

NetFlow settings are configured at the vSphere distributed switch level.\\

If you don't select ``Allow normal IO on destination ports'' mirrored traffic
is allowed out but not in on destination ports.

\subsubsection{Identify vNetwork Distributed Switch (vDS) capabilities}

Distributed virtual switches have increased capabilities when compared to
standard vSwitches such as security, traffic shaping, load balancing, VLAN
settings, portblocking and VMs joining VLANs on physical networks.\\

Distributed virtual switches can be 3rd-party products like the Cisco Nexus
1000V.\\

Port binding determines when and how a VM's virtual NIC is bound to a port.
It is configured at the port group level.\\

Port binding options:

\begin{description}

\item[static]
default, ports are bound at configuration time, the VM is guaranteed a port
on the distributed virtual switch even if it is powered off

\item[dynamic]
the port is bound when needed, allows you oversubscribe ports

\item[ephemeral]
no binding, behaves like the standard switch, can allocate ports up to the
maximum on the switch

\end{description}

\subsubsection{Create/Delete a vNetwork Distributed Switch}

\subsubsection{Add/Remove ESXi hosts from a vNetwork Distributed Switch}

\subsubsection{Add/Configure/Remove dvPort groups}

\subsubsection{Add/Remove uplink adapters to dvUplink groups}

\subsubsection{Create/Configure/Remove virtual adapters}

\subsubsection{Migrate virtual adapters to/from a vNetwork Standard Switch}

\subsubsection{Migrate virtual adapters to/from a vNetwork Distributed Switch}

VMs connected to standard virtual switches can be migrated to distributed
virtual switches using vCenter.

\subsubsection{Determine use case for a vNetwork Distributed Switch}
