\subsection{Secure vCenter Server and ESXi}

Enable tech support mode on the ESXi server enables the ESXi shell (a command-line interface on
the physical server console).\\

Enabling remote tech support on the ESXi server enables ssh.\\

Both are found in Troubleshooting Options on the ESXi console.

\subsubsection{Identify common vCenter Server privileges and roles}

\subsubsection{Describe how permissions are applied and inherited in vCenter Server}

\subsubsection{Configure and administer the ESXi firewall}

Select the host, Configuration tab, Security Profile.

\subsubsection{Enable/Configure/Disable services in the ESXi firewall}

\subsubsection{Enable Lockdown Mode}

Lockdown mode prevents users other than vpxuser from authenticating to a host.
This means that no configuration can be done directly against the ESXi host
using vSphere client or vSphere CLI and everything must be done through
vCenter.\\

root can still log in on the direct console when lockdown mode is enabled.

\subsubsection{Configure network security policies}

\subsubsection{View/Sort/Export user and group lists}

\subsubsection{Add/Modify/Remove permissions for users and groups on vCenter inventory objects}

\subsubsection{Create/Clone/Edit vCenter Server Roles}

\subsubsection{Add an ESXi Host to a directory service}

\subsubsection{Apply permissions to ESXi Hosts using Host Profiles}

\subsubsection{Determine the appropriate set of privileges for common tasks in vCenter Server}
