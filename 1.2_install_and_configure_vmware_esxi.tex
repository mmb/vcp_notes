\subsection{Install and Configure VMware ESXi}

ESXi is the hypervisor running on each physical server (the virtualization
OS). ESXi is a type 1 hypervisor meaning it runs directly on the hardware.\\

In vSphere 5, ESX has been completely replaced with ESXi.\\

ESXi is a thin install and doesn't have all drivers. Image builder can build
images with the required drivers.\\

ESXi requires a 64-bit CPU with Intel VT or AMD-V enabled with a minimum 2098MB RAM and a
1G NIC. vSphere 5.0 supports only CPUs that have LAHF and SAHF CPU instruction sets.

ESXi maximums:

\begin{itemize}

\item 160 logical CPUs per physical machine
\item 2TB of physical RAM
\item 512 VMs per host
\item 2048 vCPUs per host

\end{itemize}

host UEFI BIOS\\

In vSphere 5, Mac OSX 10.6 Snow Leopard can be virtualized (but only on Apple
hardware).\\

SNMP support was also improved in vSphere 5.

\subsubsection{Perform an interactive installation of ESXi}

\begin{enumerate}
\item access DCUI (Direct Console User Interface)
\item login with F2 and the root password
\item configure IP address, SM, DG, DNS server, hostname and domain
\item add to your DNS server
\item add to vCenter by DNS name
\end{enumerate}

\subsubsection{Deploy an ESXi host using Auto Deploy}

Auto Deploy can PXE boot stateless physical servers from a common image.\\

Auto Deploy and Image Builder:

\begin{itemize}
\item centrally manage stateless hardware
\item host configuration provided by answer file and/or host profiles
\item can deploy image based on subnet (install different images on different
subnets) and other criteria
\item Image Builder streamlines creation of customized installation media
\end{itemize}

Auto Deploy components:

\begin{itemize}

\item auto deploy server

\item rules to determine which image and host profile hosts will get, the
active rule set is ready to be deployed, the working rule set is for testing
rules before they go into production

\item image profile, an ISO that lets you install ESXi, vendors can provide the
images for their hardware, there are default image profiles and custom image
profiles

\end{itemize}

The answer file contains user-input policies for the host profile. It is
created when the profile is initially applied to a particular host. Any type
of user interaction during the boot process is capured in the answer file.
if any questions come up during ESXi boots that don't have answers, the host
will go into maintenance mode until they are answered.\\

Auto deploy boot process:

\begin{itemize}

\item the host PXE boots and gets an IP from DHCP

\item DHCP points the host to the TFTP server using DHCP option 66

\item the host contacts the TFTP server and downloads a gPXE configuration file
from DHCP option 67

\item the gPXE config file instructs the host to make an HTTP boot request to
the auto deploy server

\item the auto deploy server queries the rules engine for information about
the host

\item an image profile and a host profile are attached to the host based
on a rule set

\item ESXi is installed into the host RAM, is added to vCenter and is
configured

\item vCenter maintains the image profile and host profile for each host in
its database

\end{itemize}

Auto deploy requirements:

\begin{itemize}
\item DHCP server, option 66 is the hosname or IP of the TFTP server, option
67 is \texttt{undionly.kxpe.vmw-hardwired}
\item possible router configuration
\item PXE
\item TFTP server
\end{itemize}

Auto deploy installation and configuration tasks:

\begin{itemize}
\item install and configure auto deploy
\item install the TFTP server
\item configure DHCP
\item configure DNS
\item install and configure the Dump Collector
\item install and configure the syslog server
\end{itemize}

Auto deploy setup requirements:

\begin{itemize}
\item vCenter 5 setup files
\item VMware PowerCLI
\item Windows Server 2008 DHCP and DNS
\item ESXi 5 offline bundle
\item a host capable of running esxi5
\end{itemize}

PowerCLI is required to set up auto-deploy. It can't be done in vCenter.\\

ESXi offline bundle is imported into auto deploy repository using PowerCLI.

\begin{verbatim}
Add-EsxSoftwareDepot
Get-EsxImageProfile
New-DeployRule -Name foo -Item vib, "Test" -Pattern "mac=11:22:33:44:55:66"  
\end{verbatim}

Add rule to active set rules:

\begin{verbatim}
Add-DeployRule -DeployRule Test
Copy-DeployRule -DeployRule Test -ReplaceItem Production
Get-VMHost -Name esxi.test.com
$tr = Test-DeployRuleSetCompliance esxi.test.com
Repair-DeployRuleSetCompliance $tr
\end{verbatim}

On auto-deployed hosts, the syslog server and dump collector can be set up
using host profiles.\\

Logs viewable in ESXi DCUI:

\begin{itemize}
\item syslog
\item vmkernel
\item config
\item management agent (hostd)
\item virtualcenter agent (vpxa)
\item VMware ESXi Bbservation log (vobd)
\end{itemize}

\subsubsection{Configure NTP on an ESXi Host}

To configure NTP on a host using vSphere Client go to Host / Configuration /
Time Configuration.

\subsubsection{Configure DNS and Routing on an ESXi Host}

\subsubsection{Enable/Configure/Disable hyperthreading}

\subsubsection{Enable/Size/Disable memory compression cache}

Memory pages are compressed to 2K and stored in a per-VM compression cache.\\

Memory pages that are candidates for swap to disk are compressed.\\

Only takes place during contention.

\subsubsection{License an ESXi host}
