\subsection{Install and Configure VMware ESXi}

ESXi is the hypervisor running on each physical server (the virtualization
OS). ESXi is a type 1 hypervisor meaning it runs directly on the hardware.\\

In vSphere 5, ESX has been completely replaced with ESXi.\\

ESXi is a thin install and doesn't have all drivers. Image builder can build
images with the required drivers.\\

ESXi requires a 64-bit CPU with Intel VT or AMD-V enabled with a minimum 2098MB RAM and a
1G NIC. vSphere 5.0 supports only CPUs that have LAHF and SAHF CPU instruction sets.

ESXi maximums:

\begin{itemize}

\item 160 logical CPUs per physical machine
\item 2TB of physical RAM
\item 512 VMs per host
\item 2048 vCpus per host

\end{itemize}

host UEFI BIOS\\

In vSphere 5, Mac OSX 10.6 Snow Leopard can be virtualized (but only on Apple
hardware).\\

SNMP support was also improved in vSphere 5.

\subsubsection{Perform an interactive installation of ESXi}

\begin{enumerate}
\item access DCUI (Direct Console User Interface)
\item login with F2 and the root password
\item configure IP address, SM, DG, DNS server, hostname and domain
\item add to your DNS server
\item add to vCenter by DNS name
\end{enumerate}

\subsubsection{Deploy an ESXi host using Auto Deploy}

Auto Deploy can PXE boot stateless physical servers from a common image.\\

Auto Deploy and Image Builder:

\begin{itemize}
\item centrally manage stateless hardware
\item host configuration provided by answer file and/or host profiles
\item can deploy image based on subnet (put different types of hosts on
different subnets)
\item image builder streamlines creation of customized installation media
\end{itemize}

The answer file contains user-input policies for the host profile. It is
created when the profile is initially applied to a particular host.\\

It is recommended to set the remove syslog server for auto deployed hosts
in a host profile.

\subsubsection{Configure NTP on an ESXi Host}

To configure NTP on a host using vSphere Client go to Host / Configuration /
Time Configuration.

\subsubsection{Configure DNS and Routing on an ESXi Host}

\subsubsection{Enable/Configure/Disable hyperthreading}

\subsubsection{Enable/Size/Disable memory compression cache}

\subsubsection{License an ESXi host}
