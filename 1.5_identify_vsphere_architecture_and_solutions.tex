\subsection{Identify vSphere Architecture and Solutions}

\subsubsection{Identify available vSphere editions and features}

\begin{description}

\item[vSphere Hypervisor]
free, cannot be used with vCenter, limited features

\item[vSphere Essentials kit]
3 hosts with two CPUs each, maximum pooled vRAM of 192GB, includes vCenter

\item[vSphere Essentials Plus kit]
adds vMotion, HA, Data Recovery

\item[vSphere Acceleration kits]
standard, enterprise and enterprise plus

\item[vSphere]
ESXi per CPU socket and vRAM entitlement (does not include vCenter), standard,
enterprise and enterprise plus (advanced is no longer available)

\end{description}

Enterprise Plus is required for many new vSphere 5 features such as auto deploy,
and storage DRS.

\subsubsection{Explain ESXi and vCenter Server architectures}

vCenter Server Blueprint\\

Core Services

\begin{itemize}
\item VM provisioning
\item Task Scheduler
\item Events Logging
\item Host and VM Configuration
\item Inventory
\item vApp
\item Alarms and Events
\item Statistics and Logging
\end{itemize}

Distributed Services

\begin{itemize}
\item vMotion
\item HA
\item DRS
\end{itemize}

Additional Services

\begin{itemize}
\item Plugins like Update Manager
\item vShield Zones
\item Orchestrator
\item Data Recovery
\item Storage Monitoring
\item Hardware and Service Status
\end{itemize}

Database Interface - database connectivity\\

ESXi Host Management\\

Active Directory Integration\\

vSphere API - SDK for developers\\

To see the services associated with vCenter in the vSphere Client go to
Home / Administration / vCenter Service Status.

\subsubsection{Explain Private/Public/Hybrid cloud concepts}

\subsubsection{Determine appropriate vSphere edition based on customer requirements}
