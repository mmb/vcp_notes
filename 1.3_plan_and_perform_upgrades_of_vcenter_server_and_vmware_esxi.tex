\subsection{Plan and Perform Upgrades of vCenter Server and VMware ESXi}

Before doing an upgrade from vSphere 4 to 5, verify that your licenses will
upgrade to the level of vSphere that you need. Pay special attention to vRAM.
Do adequate planning and review the upgrade guide to identify which upgrade
scenario fits.\\

Upgrading to vCenter 5:

\begin{itemize}

\item requires vCenter downtime

\item no production downtime (no VM downtime)

\item database schema will be upgraded

\item upgrade to new vCenter Server

\item backup databasse

\item check ODBC credentials

\item uninstall Update Manager and Converter extensions (for each, uninstall
both server side and client side)

\item check vSphere compatibility matrices

\end{itemize}

Upgrading VMware Update Manager:

\begin{itemize}
\item backup database
\item check ODBC credentials
\item uninstall Update Manager client extensions
\end{itemize}

\subsubsection{Identify upgrade requirements for ESXi hosts}

Host hardware must be compatible with vSphere 5.

\subsubsection{Identify steps required to upgrade a vSphere implementation}

\begin{enumerate}
\item run vCenter host agent pre-upgrade checker
\item upgrade vCenter server
\item upgrade vPphere client
\item upgrade licensing
\item upgrade Update Manager
\item upgrade ESX/ESXi hosts
\item upgrade VMware tools and VM hardware
\item upgrade datastore to VMF5
\end{enumerate}

\subsubsection{Upgrade a vNetwork Distributed Switch}

\subsubsection{Upgrade from VMFS3 to VMFS5}

\subsubsection{Upgrade VMware Tools}

Updating VMware tools can be automated with VMware Update Manager or done
manually.

\subsubsection{Upgrade Virtual Machine hardware}

Upgrading virtual machine hardware requires VM downtime. In vSphere 5 VMs
can be updated to virtual hardware version 8. The process can be automated
using Update Manager or done manually.

\subsubsection{Upgrade an ESXi Host using vCenter Update Manager}

\begin{itemize}
\item requires host downtime
\item host must be in maintenance mode
\item use the host upgrade utility GUI
\item takes 15 - 25 minutes
\end{itemize}

If upgrading an ESX host that has a local VMFS datastore, you will be asked
if you want to preserve or overwrite it.\\

if your ESX host has custom VIBs, migration could cause problems and you may
need to build a new VIB using Image Builder.

\subsubsection{Determine whether an in-place upgrade is appropriate in a given upgrade scenario}
