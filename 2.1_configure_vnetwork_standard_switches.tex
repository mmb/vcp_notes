\subsection{Configure vNetwork Standard Switches}

Virtual networking features of ESXi can be used to build an IP network that
integrates seamlessly with the physical server environment.

\subsubsection{Identify vNetwork Standard Switch (vSS) capabilities}

vSSs are logical objects that reside in the vmkernel of an ESXi host. Each
virtual NIC connected to a virtual switch will have its own MAC address.
One or more physical NICs in the host are bound to the virtual switch to give
VMs connectivity to the outside world.\\

Virtual switches can contain one or more port groups.\\

The default number of logical ports for a standard switch is 120. When the
number of ports on an existing virtual switch is changed, the ESXi host must be
rebooted.\\

A virtual switch mimics a physical switch. vSSs operate at layer 2 and can
provide vlan tagging, security, checksums, and segmentation offload units.\\

Virtual switches are similar to physical switches:

\begin{itemize}
\item maintain MAC address tables
\item look up each frame's destination MAC on arrival
\item forward frames to one or more ports
\item  avoid unncessary deliveries
\end{itemize}

Virtual switches are different from physical switches:

\begin{itemize}
\item cannot be connected to other virtual switches the way physical switches can
\item do not require spanning tree protocol (STP)
\item isolation prevents switching loops
\item forwarding data table is unique to each virtual
\end{itemize}

Type of vSSs:

\begin{description}

\item[internal only]
used for communication of two VMs on the same ESXi host

\item[single adapter]
a virtual switch bound to a single physical adapter for communication
with the physical network

\item[NIC team]
bound to two or more physical adapters used to provide redundancy and bandwidth
aggregation for communication with the physical network

\end{description}

\subsubsection{Create/Delete a vNetwork Standard Switch}

\subsubsection{Add/Configure/Remove vmnics on a vNetwork Standard Switch}

\subsubsection{Configure vmkernel ports for network services}

vmkernel ports can be used for:

\begin{itemize}
\item Fault Tolerance logging
\item IP storage (iSCSI, NFS)
\item management
\item vMotion
\end{itemize}

Each VMkernel port has an IP address configured for it or can use DHCP.\\

You can enable vMotion and IP storage for only one port group per host.\\

The label used for the vMotion network must be the same on all ESXi hosts.

\subsubsection{Add/Edit/Remove port groups on a vNetwork Standard Switch}

All portgroups in the same broadcast domain are given the same network label.
If two portgroups are not in the same broadcast domain they will have different
network labels.

\subsubsection{Determine use cases for a vNetwork Standard Switch}
