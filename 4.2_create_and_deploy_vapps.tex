\subsection{Create and Deploy vApps}

A vApp is a container, like a resource pool and contains one or more virtual
machines. A vApp can be powered on and off, suspended and cloned.\\

When powering on a vApp in a DRS cluster in manual mode, DRS performs as if
the cluster is in automatic mode  and does not generate recommendations for
the initial placement of VMs in the vApp.

\subsubsection{Identify vApp settings}

\begin{itemize}
\item shares, reservations and limits (like resource pools)
\item startup and shutdown order
\item startup delays and startup actions
\item vService dependencies
\item OVF environment properties
\item HTTP proxy server
\end{itemize}

\subsubsection{Create/Clone/Export a vApp}

The distribution format for vApp is OVF.\\

VMware Studio can be used to create vApps.\\

A vApp can contain VMs, resource pools or other vApps.

\subsubsection{Add objects to an existing vApp}

Objects can be dragged into the vApp.

\subsubsection{Edit vApp settings}

\subsubsection{Configure IP pools}

vApps have an IP allocation policy that can be:

\begin{description}

\item[fixed]
IP addresses are configured manually

\item[transient]
IP addresses are automatically allocated from configured IP pools

\item[DHCP]
a DHCP server is used to allocate IP addresses

\end{description}

An IP pool is host range within the network.

\subsubsection{Suspend/Resume a vApp}

\subsubsection{Determine when a tiered application should be deployed as a vApp}
