\subsection{Install and Configure vCenter Server}

\begin{description}

\item[vCenter Server]
centralized management console for the entire virtualization infrastructure

\item[vSphere Client]
connects to vCenter server

\item[vMA]
Linux based virtual appliance, gives service console like ESX used to

\end{description}

vCenter entities:

\begin{description}

\item[Datacenter]
storage container inside vCenter to manage ESXi hosts, resource pools, folders,
VMs, etc.

\item[VM]
virtual machine, virtualized guest operating system running inside ESXi

\item[Host]
server running ESXi

\item[Guest]
virtualized guest operating system running on a VM

\item[Datastore]
storage for VMs, virtual representation of combinations of underlying
physical storage resources, stores virtual disks, virtual memory, VM
configuration files, log files, core dumps and anything you might add such
as ISO files

\item[Folder]
way to group other entities together in a hierarchy

\item[Resource Pool]
partitions of computing and memory resources from a single host or cluster

\item[Cluster]
aggregate computing and memory resources of a group of physical x86 servers
sharing the same network and storage arrays

\item[Network]
virtual networks, connect VMs

\end{description}

vCenter has enhanced logging support in vSphere 5 (look up new features).\\

A naming convention for hosts and virtual machines is critical.\\

Inventory can be organized by physical site, company division, purpose of
infrastructure, or another system that makes sense to the organization. The
way the entities are organized will be used later for giving access to specific
users and groups.

\begin{itemize}

\item root can contain

  \begin{itemize}
  \item folder
  \item datacenter
  \end{itemize}

\item folder can contain

  \begin{itemize}
  \item folder
  \item datacenter
  \end{itemize}

\item datacenter can contain

  \begin{itemize}
  \item folder
  \item cluster
  \item host 
  \item virtual machine
  \end{itemize}

\end{itemize}

vCenter Server Settings:

\begin{itemize}
\item manage licensing
\item database calculator
\item vCenter unique id (used for linked mode)
\item SMTP server for alerts
\item SNMP receivers
\item client timeout settings
\item logging options
\item maximum number of database connections (default 50)
\end{itemize}

\subsubsection{Identify all available vCenter Server Editions}

\subsubsection{Deploy the vCenter Appliance}

\subsubsection{Install vCenter Server into a virtual machine}

\subsubsection{Size the vCenter Server database}

\subsubsection{Install additional vCenter Server Components}

\subsubsection{Install/Remove vSphere Client plug-ins}

Some vCenter server plug-ins require client plug-ins that must be installed
on each client.

\subsubsection{Enable/Disable vSphere Client plug-ins}

\subsubsection{License vCenter Server}

\subsubsection{Determine availability requirements for a vCenter Server in a given vSphere implementation}

\subsubsection{Determine use case for vSphere Client and Web Client}

The vSphere client only displays what you have licensed (the rest is grayed
out).\\

vSphere client can connect to vCenter server or directly to an ESXi server.\\

vSphere client can show different views and will remember the last view used
in a connection. It has ``back'' functionality like a web browser.\\

The vSphere Client status bar shows tasks, alarms and the currently logged in
user.\\

The vSphere web client has been improved in vSphere 5.\\

The vSphere client has keyboard shortcuts for each view (Ctrl-Shift-h goes to
the Hosts and Clusters view).\\

Advanced Search in the vShere Client can be used to search with many different
types of criteria (such as all VMs where VMware tools are out-of-date). If
using linked mode it can search across multiple vCenters.\\

Data tables in the vSphere Client can be exported to HTML, XLS, CSV, etc.\\

The vSphere client can can reports such as host summary reports and performance
reports.\\

To get rid of Getting Started tabs in vSphere Client, go to
Edit / Client Settings and uncheck show getting started tab.\\

Internet Explorer Enhanced Security can interfere with the Host / Hardware
Status tab. It can be turned off in Windows Server Manager.\\

vCenter Client be used to review system logs. A message of the day can be set
to give users important information.
