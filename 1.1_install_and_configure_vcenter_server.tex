\subsection{Install and Configure vCenter Server}

\begin{description}

\item[vCenter Server]
centralized management console for the entire virtualization infrastructure, serves as a proxy
for managing ESXi hosts and their virtual machines

\item[vSphere Client]
connects to vCenter server and ESXi hosts

\item[vMA]
Linux based virtual appliance, gives service console like ESX used to

\end{description}

vCenter is a requirement for enterprise features.\\

vCenter entities:

\begin{description}

\item[Datacenter]
storage container inside vCenter to manage ESXi hosts, resource pools, folders,
VMs, etc.

\item[VM]
virtual machine, virtualized guest operating system running inside ESXi

\item[Host]
server running ESXi

\item[Guest]
virtualized guest operating system running on a VM

\item[Datastore]
storage for VMs, virtual representation of combinations of underlying
physical storage resources, stores virtual disks, virtual memory, VM
configuration files, log files, core dumps and anything you might add such
as ISO files

\item[Folder]
way to group other entities together in a hierarchy

\item[Resource Pool]
partitions of computing and memory resources from a single host or cluster

\item[Cluster]
aggregate computing and memory resources of a group of physical x86 servers
sharing the same network and storage arrays

\item[Network]
virtual networks, connect VMs

\end{description}

vCenter has enhanced logging support in vSphere 5 (look up new features).\\

A naming convention for hosts and virtual machines is critical.\\

Inventory can be organized by physical site, company division, purpose of
infrastructure, or another system that makes sense to the organization. The
way the entities are organized will be used later for giving access to specific
users and groups.

\begin{itemize}

\item root can contain

  \begin{itemize}
  \item folder
  \item datacenter
  \end{itemize}

\item folder can contain

  \begin{itemize}
  \item folder
  \item datacenter
  \end{itemize}

\item datacenter can contain

  \begin{itemize}
  \item folder
  \item cluster
  \item host 
  \item virtual machine
  \end{itemize}

\end{itemize}

vCenter Server Settings:

\begin{itemize}
\item manage licensing
\item database calculator
\item vCenter unique id (used for linked mode)
\item SMTP server for alerts
\item SNMP receivers
\item client timeout settings
\item logging options
\item maximum number of database connections (default 50)
\end{itemize}

\subsubsection{Identify all available vCenter Server Editions}

\subsubsection{Deploy the vCenter Appliance}

vCSA is a new option for deploying vCenter in vSphere 5. It is a Linux virtual
machine optimized for running vCenter server.\\

No installation needs to be performed.\\

Pros

\begin{itemize}
\item supports all traditional vCenter server features likes DRS, SDRS, HA,
host profiles, dvSwitch,etc.
\item no Windows license is required
\item no Windows install
\item no vCenter install
\item deployment is simple and fast
\end{itemize}

Cons

\begin{itemize}
\item doesn't support SQL Server as an external database
\item linked mode doesn't work
\item vCenter Server heartbeat doesn't work
\item no single sign-on using Windows session credentials
\item vSphere Storage Appliance (VSA) is not compatible
\end{itemize}

You will need an installed ESXi server and the vSphere Client.

\begin{enumerate}

\item download the .OVF files, appliance data disk and appliance system disk
from vmware.com
\item deploy OVF in vSphere Client. Initial credentials are root/vmware
\item perform network and timezone configuration using the text menu in the
console
\item vCSA should have a static IP address and DNS host record on your DNS
server
\item Test Settings and Save Settings in the database tab of the web GUI (port
5480)
\item start vCenter Service on the vCSA VM
\item set a static IP
\item set the timezone
\item change the root password
\item configure Windows AD authentication
\item size the embedded database
\item restart the VCSA VM
\end{enumerate}

The vCSA has the vSphere web client installed and running by default.\\

You need to add permissions for an AD group for AD users to be able to login.
It does not default to the Administrators group of the domain that the vCenter
server is in like the Windows vCenter. When connecting to the vCSA with vSphere
Client you need to prefix the username with the domain when using AD.

\subsubsection{Install vCenter Server into a virtual machine}

vCenter Server can be physical or virtual. VMware best practices recommend
virtual.\\

vCenter can participate in HA and be vMotioned.\\

If vCenter goes down in an environment with a lot of hosts, to find which
server vCenter is on you need to log in to each host manually. Use an affinity
rule to avoid this.\\

vCenter heartbeat can be used when vCenter is virtual.\\

vCenter Pre-installation

\begin{itemize}
\item to be part of a domain
\item to have a static IP and hostname
\item a back-end database
\item correct time and date
\item server must be in DNS and must be resolvable from all ESXi hosts
\item should not be a domain controller
\item account installed under should be a member of administrator group, act as
part of the operating system and log on as a service
\end{itemize}

vCenter Features

\begin{itemize}
\item ESXi management
\item VM management
\item Update Manager
\item Converter Enterprise
\item templates
\item access control
\end{itemize}

vCenter Hardware and Software Requirements

\begin{itemize}
\item 2 64-bit CPU or 1 64-bit dual core CPU (2GHz or better)
\item 4G RAM (more if database is on the same server)
\item 4GB disk (more if database is on the same server)
\item 1GB or better networking
\item Windows Server 2003 Std, Ent or Datacenter 64-bit SP2, 2003 R2 Std, End, or Datacenter
64-bit SP1, 2008 Std, Ent or Datacenter 64-bit SP2, 2008 R2 Std, Ent, or Datacenter 64-bit
\item .NET 3.5 SP1 Framework
\item Windows Installer 4.5 (if using SQL 2008 R2 Express)
\end{itemize}

Database Requirements

\begin{itemize}
\item Microsoft SQL Server 2008 R2 Express (default, for demos and eval)
\item DB2 9.5 and 9.7
\item SQL Server 2005
\item SQL Server 2008 and 2008 R2
\item Oracle 10g R2 and 11g
\end{itemize}

vCenter database calculator can estimate database space required based on
number of hosts, VMs and amount of statics required. It can be found in vCenter
Server Setting from the Home Screen.

Port Requirements

\begin{itemize}
\item 22 SSH
\item 80/443 for Web access
\item 123 NTP
\item 389 for LDAP, can be changed
\item 636 for vCenter Linked Mode, can be changed
\item 902 for heartbeat, ESXi management and VM console, vCenter to vpxa on host
\item 903 vSphere Client to VM console, vCenter to vpxa on host
\item 2025 - 2250, 8042 - 8045 High Availability
\item 3260 iSCSI
\item 8000 vMotion
\item 8080 / 8443 for web services and HTTPS web services
\item 10443 vCenter Inventory Service HTTPS
\item 10109 vCenter Inventory Serice Management
\item 10111 vCenter Inventory Service Linked Mode Communications
\item 60099 web service change service notification
\end{itemize}

When using a database on another server you must set up the 64-bit DSN for that database so you
can select it when installing vCenter.

\subsubsection{Size the vCenter Server database}

\subsubsection{Install additional vCenter Server Components}

\subsubsection{Install/Remove vSphere Client plug-ins}

Some vCenter server plug-ins require client plug-ins that must be installed
on each client.

\subsubsection{Enable/Disable vSphere Client plug-ins}

\subsubsection{License vCenter Server}

The evaluation version is good for 60 days.\\

License keys can be added from Licensing on the Home screen.

\subsubsection{Determine availability requirements for a vCenter Server in a given vSphere implementation}

\subsubsection{Determine use case for vSphere Client and Web Client}

The vSphere client only displays what you have licensed (the rest is grayed
out).\\

vSphere client can connect to vCenter server or directly to an ESXi server.\\

vSphere client can show different views and will remember the last view used
in a connection. It has ``back'' functionality like a web browser.\\

The vSphere Client status bar shows tasks, alarms and the currently logged in
user.\\

The vSphere client has keyboard shortcuts for each view (Ctrl-Shift-h goes to
the Hosts and Clusters view).\\

Advanced Search in the vShere Client can be used to search with many different
types of criteria (such as all VMs where VMware tools are out-of-date). If
using linked mode it can search across multiple vCenters.\\

Data tables in the vSphere Client can be exported to HTML, XLS, CSV, etc.\\

The vSphere client can can reports such as host summary reports and performance
reports.\\

To get rid of Getting Started tabs in vSphere Client, go to
Edit / Client Settings and uncheck show getting started tab.\\

Internet Explorer Enhanced Security can interfere with the Host / Hardware
Status tab. It can be turned off in Windows Server Manager.\\

vCenter Client be used to review system logs. A message of the day can be set
to give users important information.\\

vSphere Client is downloaded directly from vmware.com. An ESXi host or vCenter
server will direct you there.\\

The vSphere web client is new in vSphere 5. The web client requires Adobe Flash.
Adobe Flash must also be installed on the server side for authorization.\\

VMware has hinted that the web client is the future vSphere administration
and that the Windows vSphere client (written in C\#) will go away.\\

Installation of vSphere web client server is done from vCenter media. It is
typically installed on the vCenter server.\\

The vSphere web client can connect to multiple vCenter servers.\\

vSphere client plugins do not work yet in the web client.
