\subsection{Create and Configure VMFS and NFS Datastores}

\subsubsection{Identify VMFS and NFS Datastore properties}

VMFS is the VMware filesystem, a specialized virtualization clustered
filesystem that provides distributed locking. VMFS can be local, iSCSI or
Fibre Channel. If NFS is being used, VMFS is not being used.\\

VMFS is optimized for VMs and high performance. It does not have a lot of
features compared to NTFS or other filesystems.\\

VMFS-5 is the latest version of VMFS. The maximum disk size for a VM is
64TB. The best practice is to format LUNs with and 8M block size?\\

Centralized storage is required for many advanced features of vSphere like
vMotion, HA, FT and DRS. Usually the central storage is a SAN.\\

Storage congestion threshold is set per datastore and is associated with
storage IO control.\\

You can connect up to 128 hosts to a VMFS datastore.\\

When configuring NFS exports for use with vCenter use the
\texttt{no\_root\_squash} option.

\subsubsection{Identify VMFS5 capabilities}

\begin{itemize}

\item no more 2TB limit on VMFS volumes, the new limit is 64TB
\item supports 2TB RDMs
\item block size always 1 MB (no choice)
\item atomic test and set (part of VAAI), locking mechanism, hardware assisted
locking
\item sub-blocks for space efficiency (8K)
\item small file support (1K)
\item space reclamation on thin provisioned LUNs (with enhanced VAAI)
\item monitoring of space when using thin provisioning (with enhanced VAAI)

\end{itemize}

\begin{description}

\item[VASA] vSphere APIs for Storage Awareness

\item[VAAI] vSphere APIs for Array Integration, support NFS now

\end{description}

\subsubsection{Create/Rename/Delete/Unmount a VMFS Datastore}

Datastore delete permanently deletes all files associated with virtual
machines on the datastore.

\subsubsection{Mount/Unmount an NFS datastore}

\subsubsection{Extend/Expand VMFS datastores}

To increase the size of a datastore, you can add a new extent or grow an
existing extent (which requires an extent with free space immediately after
it).\\

If a datastore has powered-on VMs and is 100\% full you can increase the its
capacity only from the host that the VMs are registered on.

\subsubsection{Upgrade a VMFS3 Datastore to VMFS5}

ESXi 5.0 does not support VMFS2. ESX/ESXi 3 and 4 can use VMFS2 read only.\\

VMFS5 only works with ESXi 5.0.\\

VMFS2 must be upgraded to VMFS3 before upgrading to VMFS5.\\

The upgrade to VMFS5 cannot be reverted. An upgraded VMFS5 datastore differs
from a new VMFS5 datastore. Upgraded VMFS5 datastores retain their previous
block and subblock sizes.

\subsubsection{Place a VMFS Datastore in Maintenance Mode}

\subsubsection{Select the Preferred Path for a VMFS Datastore}

ESXi fully supports SAN multipathing. Multpathing means there are multiple
paths through HBAs and FC switches to the same LUNs for redundancy.\\

When a path fails, storage I/O might pause for 30 to 60 seconds.\\

Path selection policies:

\begin{description}

\item[fixed]
the host uses the designated preferred path if it has been set, if not the
preferred path is the first working path discovered at boot time, default
policy for most active-active storage devices

\item[MRU]
most recently used, the host selects the path it used most recently, when the
path is down the host selects another path, default policy for most
active-passive storage devices

\item[round robin]
host rotates through all active paths when connecting to active-passive arrays
or through all available paths when connecting to active-active

\end{description}
 
\subsubsection{Disable a path to a VMFS Datastore}

esxcli MASK\_PATH claim rule ?
 
\subsubsection{Determine use case for multiple VMFS/NFS Datastores}

\subsubsection{Determine appropriate Path Selection Policy for a given NFS Datastore}
