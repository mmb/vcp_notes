\subsection{Administer Virtual Machines and vApps}

\subsubsection{Identify files used by virtual machines}

\subsubsection{Identify locations for virtual machine configuration files and virtual disks}

\subsubsection{Identify common practices for securing virtual machines}

\subsubsection{Hot Extend a virtual disk}

\subsubsection{Configure virtual machine options}

virtual sockets versus cores

\subsubsection{Configure virtual machine power settings}

\subsubsection{Configure virtual machine boot options}

BIOS is called ``firmware'' by VMware. When booting a VM press F2 to enter
the BIOS. It is usually hard to press it fast enough so VM / Edit Settings /
Options / Boot Options / Force BIOS Setup is used to make it boot into the
BIOS at the next boot.\\

To boot a VM from the CD drive press ESC for the boot menu during boot.\\

vSphere 5 supports EFI firmware on VMs. EFI is a BIOS alternative that offers
built in hardware drivers and a shell. It doesn't work with Windows guests
or with network boot and is usually used for Mac OSX guests.\\

To enable EFI BIOS on a VM go to VM / Edit Settings / Options / Boot Options.\\

ESXi server can be booted on a system using EFI firmware (such as Apple
hardware), allowing existing EXSi servers to be booted from hard drives, CD
drives and USB devices.

\subsubsection{Configure virtual machine troubleshooting options}

\subsubsection{Assign a Storage Policy to a virtual machine}

Profile driven storage can be used to match storage capabilities to VM storage
requirements to meet SLAs. It leverages VASA (storage APIs for array
awareness).\\

Datastores can be manually tagged if the storage vendor has not implemented
the APIs.\\

See VM Storage Profile in the vSphere client.

\subsubsection{Verify Storage Policy compliance for virtual machines}

\subsubsection{Determine when an advanced virtual machine parameter is required}

\subsubsection{Adjust virtual machine resources (shares, limits and reservations) based on virtual machine workloads)}
