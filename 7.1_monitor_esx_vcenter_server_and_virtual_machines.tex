\subsection{Monitor ESXi, vCenter Server and Virtual Machines}

\subsubsection{Describe how Tasks and Events are viewed in vCenter Server}

Tasks are initiated by users or in some cases by the system.\\

Tasks and events are available at just about every level of the inventory
and on every type of object.\\

Events record tasks and events that occur on the system, for example:

\begin{itemize}
\item alarm condition reached
\item datastore out of space
\end{itemize}

Events can be filtered, sorted and exported in vSphere Client.

\subsubsection{Identify critical performance metrics}

\subsubsection{Explain common memory metrics}

\subsubsection{Explain common CPU metrics}

\subsubsection{Explain common network metrics}

\subsubsection{Explain common storage metrics}

\subsubsection{Compare and Contrast Overview and Advanced Charts}

\subsubsection{Configure SNMP for vCenter Server}

Types of SNMP in vSphere:

\begin{itemize}
\item outbound traps (can be sent to a network management station)
\item inbound statistics gathering
\end{itemize}

vCenter does the SNMP sending, not the ESXi server.\\

vCenter has a simple SMTP configuration, server and username only.

\subsubsection{Configure Active Directory and SMTP settings for vCenter Server}

\subsubsection{Configure vCenter Server logging options}

\subsubsection{Create a log bundle}

\subsubsection{Create/Edit/Delete a Scheduled Task}

\subsubsection{Configure/View/Print/Export resource maps}

\subsubsection{Start/Stop/Verify vCenter Server service status}

\subsubsection{Start/Stop/Verify ESXi host agent status}

\subsubsection{Configure vCenter Server timeout settings}

\subsubsection{Monitor/Administer vCenter Server connections}

\subsubsection{Create an Advanced Chart}

\subsubsection{Determine host performance using resxtop and guest Perfmon}

\subsubsection{Given performance data, identify the affected vSphere resource}
